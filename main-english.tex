% !TeX spellcheck = en-US
% !TeX encoding = utf8
% !TeX program = lualatex
% !BIB program = biber
% -*- coding:utf-8 mod:LaTeX -*-


% vv  scroll down to line 200 for content  vv


\let\ifdeutsch\iffalse
\let\ifenglisch\iftrue
\input{pre-documentclass}
\documentclass[
  % fontsize=11pt is the standard
  a4paper,  % Standard format - only KOMAScript uses paper=a4 - https://tex.stackexchange.com/a/61044/9075
  twoside,  % we are optimizing for both screen and two-side printing. So the page numbers will jump, but the content is configured to stay in the middle (by using the geometry package)
  bibliography=totoc,
  %               idxtotoc,   %Index ins Inhaltsverzeichnis
  %               liststotoc, %List of X ins Inhaltsverzeichnis, mit liststotocnumbered werden die Abbildungsverzeichnisse nummeriert
  headsepline,
  cleardoublepage=empty,
  parskip=half,
  %               draft    % um zu sehen, wo noch nachgebessert werden muss - wichtig, da Bindungskorrektur mit drin
  draft=false
]{scrbook}
\input{config}

\addbibresource{bibliography.bib}
\usepackage[
  title={Evaluating Various Transaction Processing
Characteristics of Permissioned Blockchain Networks},
  author={Vikas Khinchi},
  type=Master Thesis,
  institute=iaas, % or other institute names - or just a plain string using {Demo\\Demo...}
  course={Infotech},
  examiner={Prof.\ Dr.\ Dr.\ h.\ c.\ Frank Leymann},
  supervisor={Ghareeb Falazi,\ M.Sc.},
  startdate={April 30, 2018},
  enddate={October 30, 2018}
]{scientific-thesis-cover}

% Hier stehen alle Abkürzungen
\newacronym{er}{ER}{error rate}
\newacronym{fr}{FR}{Fehlerrate}
\newacronym[plural={RDBMS},shortplural={RDBMS}]{rdbms}{RDBMS}{Relational Database Management System}
\newacronym[plural={MSPs}, shortplural={MSPs}]{msp}{MSP}{Membership Service Provider}
\newacronym[plural={DLT}, shortplural={DLT}]{dlt}{DLT}{Distributed Ledger Technology}
\newacronym{css}{CSS}{Cascading Style Sheet}
\newacronym{html}{HTML}{Hyper Text Markup Language}
\newacronym{cft}{CFT}{Crash Fault Tolerant}
\newacronym{bft}{BFT}{Byzantine Fault Tolerant}
\newacronym{poet}{PoET}{Proof of Elapsed Time}
\newacronym{pow}{PoW}{Proof of Work}
\newacronym{rbft}{RBFT}{ Redundant Byzantine Fault Tolerance}
\newacronym{pbft}{PBFT}{ Practical Byzantine Fault Tolerance}
\newacronym[plural={CAs}, shortplural={CAs}]{ca}{CA}{Certificate Authority}
\newacronym[plural={OUs}, shortplural={OUs}]{ou}{OU}{Organizational Unit}
\newacronym{escc}{ESCC}{Endorsement System Chaincode}
\newacronym{vscc}{VSCC}{Validation System Chaincode}
\newacronym{mvcc}{MVCC}{MultiVersion Concurrency Control}
\newacronym{utxo}{UTXO}{Unspent Transaction Output}
\newacronym{dos}{DoS}{Denial of Service}
\newacronym[plural={OSNs},shortplural={OSNs}]{osn}{OSN}{Ordering Service Node}
\newacronym{ptm}{PTM}{Peer Transaction Manager}


\makeindex

\begin{document}

%tex4ht-Konvertierung verschönern
\iftex4ht
  % tell tex4ht to create picures also for formulas starting with '$'
  % WARNING: a tex4ht run now takes forever!
  \Configure{$}{\PicMath}{\EndPicMath}{}
  %$ % <- syntax highlighting fix for emacs
  \Css{body {text-align:justify;}}

  %conversion of .pdf to .png
  \Configure{graphics*}
  {pdf}
  {\Needs{"convert \csname Gin@base\endcsname.pdf
      \csname Gin@base\endcsname.png"}%
    \Picture[pict]{\csname Gin@base\endcsname.png}%
  }
\fi

%\VerbatimFootnotes %verbatim text in Fußnoten erlauben. Geht normalerweise nicht.

\input{commands}
\pagenumbering{arabic}
\Titelblatt

%Eigener Seitenstil fuer die Kurzfassung und das Inhaltsverzeichnis
\deftripstyle{preamble}{}{}{}{}{}{\pagemark}
%Doku zu deftripstyle: scrguide.pdf
\pagestyle{preamble}
\renewcommand*{\chapterpagestyle}{preamble}



%Kurzfassung / abstract
%auch im Stil vom Inhaltsverzeichnis
\ifdeutsch
  \section*{Kurzfassung}
\else
  \section*{Abstract}
\fi
Blockchain is a ditributed ledger technology that consists of peers holding the same copy of data and it eliminates the need for a third party in exchanging assets and performing business transactions. In permissionless blockchain, any entity can participate in submitting and validationg a transaction as the entire blockchain network is public. Example of permissionless blockchain includes Ethereum and Bitcoin network. On the other hand, permissioned blockchains only allow certain entitites that have the right permission to be included in the blockchain network and to participate in the transactions proposal, execution and validation stages. Hyperledger Fabric is one such example involving a completely modular approach. It provides privacy and confidentiality in the blockchain network by using the concept of channels and it uses chaincode for transaction execution. \linebreak \linebreak
The main aim of this thesis is to investigate the transactional properties of permissioned blockchain networks like
Hyperledger Fabric and evaluate them by comparing them to the ACID properties of RDBMS. A supply chain application has been developed that demonstrates the various business transactions like getting product details, creating a new product and updating a product. Hyperledger Fabric performs business transactions with the help of chaincodes that are deployed on peers and channels are introduced to provide privacy. A web frontend has been created that serves as a 2-P transaction monitor for writing data to different channels. Also an android application is created that gets all the products, creates products and updates existing products in the supply chain. Rollback, savepoint and nested transactions are important part of any business transactions and hence, an equivalent system that implements these properties with blockchain is developed. Also the immutability of blockchain involving number of peers is studied and it is found that the blockchain itself is immutable but the state database on a single peer can be modified in absence of efficient endorsement policy. 

\cleardoublepage


% BEGIN: Verzeichnisse

\iftex4ht
\else
  \microtypesetup{protrusion=false}
\fi

%%%
% Literaturverzeichnis ins TOC mit aufnehmen, aber nur wenn nichts anderes mehr hilft!
% \addcontentsline{toc}{chapter}{Literaturverzeichnis}
%
% oder zB
%\addcontentsline{toc}{section}{Abkürzungsverzeichnis}
%
%%%

%Produce table of contents
%
%In case you have trouble with headings reaching into the page numbers, enable the following three lines.
%Hint by http://golatex.de/inhaltsverzeichnis-schreibt-ueber-rand-t3106.html
%
\makeatletter
\renewcommand{\@pnumwidth}{2em}
\makeatother
%
\tableofcontents

% Bei einem ungünstigen Seitenumbruch im Inhaltsverzeichnis, kann dieser mit
% \addtocontents{toc}{\protect\newpage}
% an der passenden Stelle im Fließtext erzwungen werden.

\listoffigures
\listoftables

%Wird nur bei Verwendung von der lstlisting-Umgebung mit dem "caption"-Parameter benoetigt
%\lstlistoflistings 
%ansonsten:
\ifdeutsch
  \listof{Listing}{Verzeichnis der Listings}
\else
  \listof{Listing}{List of Listings}
\fi

%mittels \newfloat wurde die Algorithmus-Gleitumgebung definiert.
%Mit folgendem Befehl werden alle floats dieses Typs ausgegeben
\ifdeutsch
  \listof{Algorithmus}{Verzeichnis der Algorithmen}
\else
  \listof{Algorithmus}{List of Algorithms}
\fi
%\listofalgorithms %Ist nur für Algorithmen, die mittels \begin{algorithm} umschlossen werden, nötig

% Abkürzungsverzeichnis
\printnoidxglossaries

\iftex4ht
\else
  %Optischen Randausgleich und Grauwertkorrektur wieder aktivieren
  \microtypesetup{protrusion=true}
\fi

% END: Verzeichnisse


% Headline and footline
\renewcommand*{\chapterpagestyle}{scrplain}
\pagestyle{scrheadings}
\pagestyle{scrheadings}
\ihead[]{}
\chead[]{}
\ohead[]{\headmark}
\cfoot[]{}
\ofoot[\usekomafont{pagenumber}\thepage]{\usekomafont{pagenumber}\thepage}
\ifoot[]{}


%% vv  scroll down for content  vv %%































%%%%%%%%%%%%%%%%%%%%%%%%%%%%%%%%%%%%%%%%%%%%%%%%%%%%%%%%%%%%%%%%%%%%%%%%%%%%%%
%
% Main content starts here
%
%%%%%%%%%%%%%%%%%%%%%%%%%%%%%%%%%%%%%%%%%%%%%%%%%%%%%%%%%%%%%%%%%%%%%%%%%%%%%%


\chapter{Introduction}

ACID defines the transaction properties in Relational Database Management Systems that help to achieve reliability in business transactions. The main focus is on achieving database consistency. The significance of the ACID properties is described as follows :
\begin{enumerate}
  \item Atomicity : The transaction is considered as an atomic unit of work and a transaction can either be committed successfully in the database or it can be aborted due to a failure but it can never be partially committed. There are two main operations \textbf{Abort} and \textbf{Commit}. It is also called as \verb|`All or nothing`| property \cite{RVvdA2016}.
  \item The numbers starts at 1 with every call to the enumerate environment.
\end{enumerate}

\chapter{Chapter Two}
\label{chap:k2}



\blinddocument

\chapter{vikas}

\chapter{Conclusion and Outlook}
\label{chap:zusfas}

\section*{Outlook}

\printbibliography

\appendix
%\input{latexhints-english}

\pagestyle{empty}
\renewcommand*{\chapterpagestyle}{empty}
\Versicherung

\end{document}
